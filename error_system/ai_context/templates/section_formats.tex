% Section Formatting Collection
% Various section header and content formatting styles

% ====================
% Section Style 1: Bold with Horizontal Rule
% ====================
\newcommand{\sectionone}[1]{%
  \vspace{0.3cm}
  {\large\bfseries #1}\\
  \vspace{0.1cm}
  \hrule height 0.5pt
  \vspace{0.2cm}
}

% Usage:
% \sectionone{EXPERIENCE}


% ====================
% Section Style 2: Colored with Thick Line
% ====================
\newcommand{\sectiontwo}[1]{%
  \vspace{0.3cm}
  {\large\bfseries\color{blue!70!black} #1}\\[0.05cm]
  {\color{blue!60}\hrule height 2pt}
  \vspace{0.2cm}
}

% Usage:
% \sectiontwo{EDUCATION}


% ====================
% Section Style 3: Uppercase with Bottom Border
% ====================
\newcommand{\sectionthree}[1]{%
  \vspace{0.3cm}
  {\large\bfseries\MakeUppercase{#1}}
  \vspace{0.1cm}
  \hrule height 1pt
  \vspace{0.2cm}
}

% Usage:
% \sectionthree{Skills}


% ====================
% Section Style 4: Colored Background Box
% ====================
\newcommand{\sectionfour}[1]{%
  % Requires: tikz, xcolor
  \vspace{0.3cm}
  \begin{tikzpicture}
    \node[rectangle, fill=blue!20, minimum width=\textwidth,
          minimum height=0.8cm, anchor=west, font=\large\bfseries] at (0,0)
          {\hspace{0.2cm}#1};
  \end{tikzpicture}
  \vspace{0.2cm}
}

% Usage:
% \sectionfour{PROJECTS}


% ====================
% Section Style 5: Left Accent Bar
% ====================
\newcommand{\sectionfive}[1]{%
  % Requires: tikz, xcolor
  \vspace{0.3cm}
  \begin{tikzpicture}
    \fill[blue!60] (0,0) rectangle (0.2cm, 0.6cm);
    \node[anchor=west, font=\large\bfseries] at (0.4cm, 0.3cm) {#1};
  \end{tikzpicture}
  \vspace{0.2cm}
}

% Usage:
% \sectionfive{CERTIFICATIONS}


% ====================
% Section Style 6: Centered with Lines
% ====================
\newcommand{\sectionsix}[1]{%
  \vspace{0.3cm}
  \begin{center}
    \begin{tikzpicture}
      \draw[gray, line width=1pt] (0,0) -- (2,0);
      \node at (3.5,0) {\large\bfseries #1};
      \draw[gray, line width=1pt] (5,0) -- (7,0);
    \end{tikzpicture}
  \end{center}
  \vspace{0.2cm}
}

% Usage:
% \sectionsix{PUBLICATIONS}


% ====================
% Section Style 7: Numbered Sections
% ====================
\newcounter{sectioncounter}
\newcommand{\sectionseven}[1]{%
  \stepcounter{sectioncounter}
  \vspace{0.3cm}
  {\large\bfseries\color{blue!70!black}\thesectioncounter. #1}\\
  \vspace{0.1cm}
  {\color{gray!50}\hrule height 0.5pt}
  \vspace{0.2cm}
}

% Usage:
% \sectionseven{Professional Experience}


% ====================
% Section Style 8: Minimal with Color
% ====================
\newcommand{\sectioneight}[1]{%
  \vspace{0.3cm}
  {\Large\bfseries\color{purple!70!black} #1}
  \vspace{0.2cm}
}

% Usage:
% \sectioneight{AWARDS}


% ====================
% Experience Entry Style 1: Standard
% ====================
\newcommand{\experienceone}[4]{%
  % #1: Job title, #2: Date, #3: Company, #4: Location
  \textbf{#1} \hfill \textit{#2}\\
  \textit{#3} \hfill #4\\
  \vspace{0.1cm}
}

% Usage:
% \experienceone{Software Engineer}{Jan 2020 -- Present}{Tech Company}{San Francisco, CA}


% ====================
% Experience Entry Style 2: Bold Company
% ====================
\newcommand{\experiencetwo}[4]{%
  % #1: Company, #2: Job title, #3: Date, #4: Location
  \textbf{#1} \hfill #3\\
  \textit{#2} \hfill \textit{#4}\\
  \vspace{0.1cm}
}

% Usage:
% \experiencetwo{Google LLC}{Senior Software Engineer}{2021 -- Present}{Mountain View, CA}


% ====================
% Experience Entry Style 3: Colored Elements
% ====================
\newcommand{\experiencethree}[4]{%
  % #1: Job title, #2: Company, #3: Date, #4: Location
  {\color{blue!80!black}\textbf{#1}} \hfill {\color{gray}#3}\\
  #2 \hfill #4\\
  \vspace{0.1cm}
}

% Usage:
% \experiencethree{Product Manager}{Amazon}{2019 -- 2021}{Seattle, WA}


% ====================
% Education Entry Style 1: Degree First
% ====================
\newcommand{\educationone}[4]{%
  % #1: Degree, #2: Field, #3: University, #4: Year
  \textbf{#1 in #2} \hfill #4\\
  \textit{#3}\\
  \vspace{0.15cm}
}

% Usage:
% \educationone{Bachelor of Science}{Computer Science}{MIT}{2015 -- 2019}


% ====================
% Education Entry Style 2: University First
% ====================
\newcommand{\educationtwo}[5]{%
  % #1: University, #2: Degree, #3: Field, #4: Year, #5: GPA
  \textbf{#1} \hfill #4\\
  #2 in #3\\
  GPA: #5\\
  \vspace{0.15cm}
}

% Usage:
% \educationtwo{Stanford University}{Master of Science}{Artificial Intelligence}{2019 -- 2021}{3.9/4.0}


% ====================
% Skill Category Style 1: Bold Header with List
% ====================
\newcommand{\skillcategory}[2]{%
  % #1: Category name, #2: Skills list
  \textbf{#1:} #2\\
  \vspace{0.1cm}
}

% Usage:
% \skillcategory{Programming Languages}{Python, JavaScript, Java, C++}


% ====================
% Skill Category Style 2: Two Column Layout
% ====================
\newcommand{\skilltwocolumn}[2]{%
  \begin{minipage}[t]{0.48\textwidth}
    #1
  \end{minipage}
  \hfill
  \begin{minipage}[t]{0.48\textwidth}
    #2
  \end{minipage}
  \vspace{0.2cm}
}

% Usage:
% \skilltwocolumn{
%   \skillcategory{Frontend}{React, Vue, Angular}
% }{
%   \skillcategory{Backend}{Node.js, Django, Spring}
% }


% ====================
% Project Entry Style 1: Title with Link
% ====================
\newcommand{\projectone}[3]{%
  % #1: Project name, #2: URL, #3: Description
  \textbf{\href{#2}{#1}}\\
  #3\\
  \vspace{0.2cm}
}

% Usage:
% \projectone{E-Commerce Platform}{https://github.com/user/project}{Built scalable platform using MERN stack.}


% ====================
% Project Entry Style 2: Title with Tech Stack
% ====================
\newcommand{\projecttwo}[3]{%
  % #1: Project name, #2: Description, #3: Technologies
  \textbf{\color{blue!70!black}#1}\\
  #2\\
  \textit{Technologies: #3}\\
  \vspace{0.2cm}
}

% Usage:
% \projecttwo{AI Chatbot}{Developed chatbot using GPT-4 API}{Python, FastAPI, React}


% ====================
% Achievement/Award Entry
% ====================
\newcommand{\achievement}[3]{%
  % #1: Award name, #2: Organization, #3: Year
  \textbf{#1} \hfill #3\\
  \textit{#2}\\
  \vspace{0.15cm}
}

% Usage:
% \achievement{Employee of the Year}{Tech Company Inc.}{2023}


% ====================
% Publication Entry (Simple)
% ====================
\newcommand{\publicationsimple}[4]{%
  % #1: Authors, #2: Title, #3: Journal/Conference, #4: Year
  #1, ``#2,'' \textit{#3}, #4.\\
  \vspace{0.15cm}
}

% Usage:
% \publicationsimple{Smith, J., Doe, J.}{Novel ML Algorithm}{Journal of AI Research}{2023}


% ====================
% Certification Entry
% ====================
\newcommand{\certification}[3]{%
  % #1: Certification name, #2: Issuer, #3: Date
  \textbf{#1}\\
  #2 \hfill \textit{#3}\\
  \vspace{0.15cm}
}

% Usage:
% \certification{AWS Certified Solutions Architect}{Amazon Web Services}{June 2023}


% ====================
% Two-Column Content Layout
% ====================
\newcommand{\twocolumncontent}[2]{%
  % #1: Left column content, #2: Right column content
  \begin{minipage}[t]{0.30\textwidth}
    #1
  \end{minipage}
  \hfill
  \begin{minipage}[t]{0.65\textwidth}
    #2
  \end{minipage}
}

% Usage:
% \twocolumncontent{
%   Left sidebar content (skills, contact, etc.)
% }{
%   Main content (experience, education, etc.)
% }


% ====================
% Timeline Entry (with visual element)
% ====================
\newcommand{\timelineentry}[3]{%
  % #1: Date, #2: Title, #3: Description
  % Requires: tikz
  \begin{tikzpicture}
    \fill[blue!60] (0,0.2) circle (0.1cm);
    \draw[blue!60, line width=1pt] (0,0.2) -- (0.3,0.2);
    \node[anchor=west] at (0.4,0.4) {\textbf{#2}};
    \node[anchor=west, color=gray] at (0.4,0) {\textit{#1}};
  \end{tikzpicture}\\
  #3\\
  \vspace{0.2cm}
}

% Usage:
% \timelineentry{Jan 2020}{Started at Company}{Led development of key features.}


% ====================
% Sidebar Section (for two-column layouts)
% ====================
\newcommand{\sidebarsection}[1]{%
  \vspace{0.3cm}
  {\large\bfseries\color{white} #1}\\
  {\color{blue!40}\hrule height 1pt}
  \vspace{0.2cm}
}

% Usage (within sidebar):
% \sidebarsection{SKILLS}


% ====================
% Contact Info Block
% ====================
\newcommand{\contactblock}[4]{%
  % #1: Email, #2: Phone, #3: LinkedIn, #4: GitHub
  \begin{tabular}{l l}
    \textbf{Email:} & \href{mailto:#1}{#1} \\
    \textbf{Phone:} & #2 \\
    \textbf{LinkedIn:} & \href{#3}{Profile} \\
    \textbf{GitHub:} & \href{#4}{@username}
  \end{tabular}
}

% Usage:
% \contactblock{email@example.com}{555-1234}{https://linkedin.com/in/user}{https://github.com/user}


% ====================
% Horizontal Skill Bars
% ====================
\newcommand{\skillbar}[2]{%
  % #1: Skill name, #2: Proficiency (0-5)
  % Requires: tikz
  \textbf{#1}\\[-0.1cm]
  \begin{tikzpicture}
    \fill[gray!20] (0,0) rectangle (5cm, 0.3cm);
    \fill[blue!60] (0,0) rectangle (#2cm, 0.3cm);
  \end{tikzpicture}
  \vspace{0.15cm}
}

% Usage:
% \skillbar{Python}{4.5}
% \skillbar{JavaScript}{4.0}


% ====================
% Bullet Point Styling
% ====================
% Compact bullets (no spacing)
\newcommand{\compactlist}{%
  \begin{itemize}[leftmargin=*, nosep]
}

% Standard bullets (slight spacing)
\newcommand{\standardlist}{%
  \begin{itemize}[leftmargin=*, itemsep=2pt]
}

% Usage:
% \compactlist
%   \item First item
%   \item Second item
% \end{itemize}


% ====================
% Horizontal Rule Variants
% ====================
\newcommand{\thickrule}{%
  \vspace{0.1cm}
  \hrule height 1pt
  \vspace{0.1cm}
}

\newcommand{\thinrule}{%
  \vspace{0.1cm}
  \hrule height 0.3pt
  \vspace{0.1cm}
}

\newcommand{\coloredrule}[1]{%
  \vspace{0.1cm}
  {\color{#1}\hrule height 1pt}
  \vspace{0.1cm}
}

% Usage:
% \thickrule
% \coloredrule{blue!60}


% ====================
% Example Full Section
% ====================
%
% \sectionone{EXPERIENCE}
%
% \experienceone{Senior Software Engineer}{2021 -- Present}{Google}{Mountain View, CA}
% \compactlist
%   \item Led development of microservices architecture
%   \item Reduced API latency by 40\%
%   \item Mentored 5 junior engineers
% \end{itemize}
%
% \experienceone{Software Engineer}{2019 -- 2021}{Startup Inc.}{San Francisco, CA}
% \compactlist
%   \item Built full-stack application using React and Node.js
%   \item Implemented CI/CD pipeline
% \end{itemize}
